\chapter{CONCLUSION}

\hspace*{0.3in} social
media includes high-security risks as well as risks to privacy.
Because of their centralized infrastructure, their massive
archive of all the personally identifiable data a hacker could
ever need, and the general public’s ignorance of how to
properly use privacy settings to improve their online security
[9], they run this danger. There is also a huge danger, because
a lot of people, especially adolescents, always tend to trust
other people quickly. So, they become extremely confident
about others. Not only that they also share private details about
themselves without a proper understanding of what kind of
details they should share about themselves online.

   Social media have some benefits, but in addition to these
advantages, OSN's posed some similar concerns. Users'
privacy and protection, and their information, are key issues
in social media [4]. While there is some general opinion about
what social media security and what it can help the online
users, there are still many unanswered questions. We think
that the headway of new technology as a rule and specifically,
social sites will bring new security risks which may open the
doors to vindictive performers, key lumberjacks, Trojan
horses, phishing, spies, viruses and attackers [6]. However,
there are many possible solutions presented to avoid the risks.
Information security experts, government officials, and other
intelligence officers need to develop new strategies that
combat and adjust to the emerging future risks and threats [6].
Moreover in the technical aspect, for preserving the security
of social media, techniques like K-anonymity and diversity
can be used\\





\chapter{HISTORY AND EVOLUTION OF SOCIAL MEDIA}\
\hspace*{0.3in}The use of the internet started to spread and people
experiences new life since 1980 . Social media comprises
communication websites that facilitate relationship forming
between users from diverse backgrounds, resulting in a rich
social structure . More specifically  define social media
as Social media is a means of contact for online interactions
between the end users (viewers) and data generators (data
owners ) who build virtual communities using online social
networks (OSN) .

\\
\hspace*{0.3in}Its being protected by internet-connected systems, including hardware, software and data, from cyber attacks. In a computing context, security comprises cyber security and physical security both are used by enterprises to safe against unauthorized access to data centre and other computerized systems. The security, which is designed to maintain the confidentiality, integrity and availability of data,is a subset of cyber security. \\

\hspace*{0.3in}Historically, organizations and governments have taken a reactive, "point product" approach to combating cyber threats, produce something together individual security technologies - one on top of another to safe their networks andthe valuable data within them. Not only is this method expensive and complex, butnews of damaging cyber breaches continues to dominate headlines, rendering this method ineffective. In fact, given the area of group of people of data breaches, the topic of cyber security has launched to the top of the priority list for boards of directors, which they seeked as far as less risky way. Instead, organizations can a natively integrated, automated Next-Generation Security Platform that is specifically designed to provide consistent, prevention-based protection - on theendpoint, in the data centre, on the network, in public and private clouds, and acrossSaabs environments. By focusing on prevention, organizations can prevent cyber threats from impacting the network in the first place, and less overall cyber securityrisk to a manageable degree \\

\hspace*{0.3in}The Cybersecurity checking began in the 1970s when researcher Bob Thomas created a computer program called Creeper that could move across ARPANET's network. Ray Tomlinson, the innovator of email, wrote the program Reaper, which chased and deleted Creepers. Reaper was the very first example of checking a malware antivirus software and the first self-replicating program i.e. Viruses, as it made first-ever computer worms and trojans.
\\

\hspace*{0.3in} In 1971s, Programmer Bob Thomas made history by innovating a program that is widely accepted as the first incident ever computer trojan as the worm and trojan bounced between computers pc,which has groundbreaker at the time. The trojan was not at all malicious.Manke and Winkler the measures that create the greatest likelihood of security awareness success include the use of creativity in disseminating materials and participatory experiences.

Per NIST 800-50, an awareness program, unlike a security training program, specifically intends to change behavior and culture. It aims to provide information that impacts daily actions. That requires a drastically different approach than just providing information. While employees in some organizations get specific security training, for a vast majority of technology users (common people including college students and school kids), security awareness is limited to some tips for security available in some websites. This is why most security awareness programs fail. The increasing number of data breaches and other cyber-attacks clearly demonstrate that these tips are not enough to raise public security awareness to a level required to create a secure cyber culture.

project and the results of the pilot study that has been done. We have organized the paper in the following way. In next section, we describe related work on this topic. In section 3, we describe the project in details. In section 4 results of the pilot study are discussed. Finally, we draw the conclusion from our study and state our direction for future research on this topic.

    Users exchange a huge amount of personal details on
social networks, making them a target for different types of
Internet attacks, including identity theft, phishing, cyber
bullies, spamming, Web fraud, etc. Social networks provide
hackers with vast opportunities to rob identity. In these types
of attacks, a malicious individual may steal his or her personal
details, including bank accounts, addresses, telephone
numbers, etc., without the user's permission, and using it to
commit cyber-crime. For example, a lot of social networks,
including Facebook, give their users game apps [3]. 

   To complete the registration process, such applications
include personal details, like the user’s credit card details,
phone number, email, etc. Of course, when a user shares the
phone number and credit card details the risk of personal
details theft and phishing attacks is increased. In certain cases,
apps that result in the user resorting to redirect the user's
attention to harmful content and damage its credibility.
   Some of the most obvious potentially innocuous
possibilities in the sense of social networking may be the
illegal use for promotional purposes of personal details, the
collection of possible friends or the discovery of content that
may be of interest. Such techniques are considered a common
process within social networks, and everybody knows about
the collection, review, and usage of personal information for
various purposes, including commercial usage. For one thing,
it has already verified the transfer of personal data from
different social networks.
     One of the main issues for users is that numerous user
specific data leakage can be observed as a consequence of the
social network's failure within the framework of various
initiatives. One causes of significant disruption is hacking user
accounts or lack of accountability, and intercepting all
personal information. When the problem is huge, there will be
more serious issues. There are several possible risks to users,
like computer bugs, malware, Trojan horse, phishing, and
other malicious software, and they can be used to steal
sensitive information from the user.
    According to experts, phishing attacks are one of the most
common cybercrime attacks and the key focus is Internet
payments, Internet banking, Internet stocks, online games,
Web 2.0 technology used pages, and so on [3]. Beyond the
danger of misuse of personal details, social networks are an
instrument for mass demonstrations in the sense of threats to
public security. The disruptive problems of social networks
are revealed to outside intervention, creating tensions between
the government and the people, demonstrations in a short time
\\
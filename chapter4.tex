\chapter{AI-ASSISTED QUANTUM SYSTEM DESIGN \& CONTROL}
\hspace*{0.3in}Artificial Intelligence (AI) is playing an increasingly critical role in the optimization and stability of quantum computing systems. The delicate nature of qubits—susceptible to noise, drift, and hardware imperfections—demands continuous calibration and intelligent control strategies. AI-driven methods, particularly reinforcement learning (RL), have shown promise in addressing these challenges in ways that scale beyond traditional manual approaches.
\section{AI for Calibration, Pulse Shaping, and Continuous Recalibration}
\hspace*{0.3in}Quantum devices require precise control over gate operations and qubit interactions. Conventional calibration procedures, often manual and time-consuming, become impractical as qubit counts scale. AI-based approaches automate these processes by learning the relationships between control parameters and system performance. For example, reinforcement learning agents can optimize pulse sequences to minimize error rates while adapting dynamically to hardware drift. Continuous recalibration ensures that quantum processors remain operationally stable during extended computations, reducing downtime and increasing fidelity.
\section{RL and Model-Free Control Loops (Practical Examples)}
\hspace*{0.3in}Model-free RL techniques are particularly valuable in scenarios where physical models of noise and hardware imperfections are incomplete. Recent studies demonstrate how RL agents can iteratively adjust gate parameters without explicit knowledge of the underlying system [4]. In practice, this allows devices to autonomously maintain performance, even in environments where fluctuations occur unpredictably. Such closed-loop learning reduces the reliance on human expertise and paves the way for “self-healing” quantum hardware.
\section{Qubit Routing, Compilation-Aware Optimization}
\hspace*{0.3in}As quantum algorithms are mapped onto hardware, qubits must often be swapped or routed due to limited connectivity in quantum processors. This introduces additional gates, which amplify error rates. AI-assisted compilation tools—leveraging deep reinforcement learning—can generate noise-adaptive routing strategies, cutting down on unnecessary operations and improving overall success probability [5]. By combining hardware-aware compilation with real-time calibration, these approaches enhance the efficiency of near-term quantum devices and bring hybrid quantum-classical workflows closer to practical deployment.
